% Options for packages loaded elsewhere
% Options for packages loaded elsewhere
\PassOptionsToPackage{unicode}{hyperref}
\PassOptionsToPackage{hyphens}{url}
\PassOptionsToPackage{dvipsnames,svgnames,x11names}{xcolor}
%
\documentclass[
  letterpaper,
  DIV=11,
  numbers=noendperiod]{scrreprt}
\usepackage{xcolor}
\usepackage{amsmath,amssymb}
\setcounter{secnumdepth}{5}
\usepackage{iftex}
\ifPDFTeX
  \usepackage[T1]{fontenc}
  \usepackage[utf8]{inputenc}
  \usepackage{textcomp} % provide euro and other symbols
\else % if luatex or xetex
  \usepackage{unicode-math} % this also loads fontspec
  \defaultfontfeatures{Scale=MatchLowercase}
  \defaultfontfeatures[\rmfamily]{Ligatures=TeX,Scale=1}
\fi
\usepackage{lmodern}
\ifPDFTeX\else
  % xetex/luatex font selection
\fi
% Use upquote if available, for straight quotes in verbatim environments
\IfFileExists{upquote.sty}{\usepackage{upquote}}{}
\IfFileExists{microtype.sty}{% use microtype if available
  \usepackage[]{microtype}
  \UseMicrotypeSet[protrusion]{basicmath} % disable protrusion for tt fonts
}{}
\makeatletter
\@ifundefined{KOMAClassName}{% if non-KOMA class
  \IfFileExists{parskip.sty}{%
    \usepackage{parskip}
  }{% else
    \setlength{\parindent}{0pt}
    \setlength{\parskip}{6pt plus 2pt minus 1pt}}
}{% if KOMA class
  \KOMAoptions{parskip=half}}
\makeatother
% Make \paragraph and \subparagraph free-standing
\makeatletter
\ifx\paragraph\undefined\else
  \let\oldparagraph\paragraph
  \renewcommand{\paragraph}{
    \@ifstar
      \xxxParagraphStar
      \xxxParagraphNoStar
  }
  \newcommand{\xxxParagraphStar}[1]{\oldparagraph*{#1}\mbox{}}
  \newcommand{\xxxParagraphNoStar}[1]{\oldparagraph{#1}\mbox{}}
\fi
\ifx\subparagraph\undefined\else
  \let\oldsubparagraph\subparagraph
  \renewcommand{\subparagraph}{
    \@ifstar
      \xxxSubParagraphStar
      \xxxSubParagraphNoStar
  }
  \newcommand{\xxxSubParagraphStar}[1]{\oldsubparagraph*{#1}\mbox{}}
  \newcommand{\xxxSubParagraphNoStar}[1]{\oldsubparagraph{#1}\mbox{}}
\fi
\makeatother


\usepackage{longtable,booktabs,array}
\usepackage{calc} % for calculating minipage widths
% Correct order of tables after \paragraph or \subparagraph
\usepackage{etoolbox}
\makeatletter
\patchcmd\longtable{\par}{\if@noskipsec\mbox{}\fi\par}{}{}
\makeatother
% Allow footnotes in longtable head/foot
\IfFileExists{footnotehyper.sty}{\usepackage{footnotehyper}}{\usepackage{footnote}}
\makesavenoteenv{longtable}
\usepackage{graphicx}
\makeatletter
\newsavebox\pandoc@box
\newcommand*\pandocbounded[1]{% scales image to fit in text height/width
  \sbox\pandoc@box{#1}%
  \Gscale@div\@tempa{\textheight}{\dimexpr\ht\pandoc@box+\dp\pandoc@box\relax}%
  \Gscale@div\@tempb{\linewidth}{\wd\pandoc@box}%
  \ifdim\@tempb\p@<\@tempa\p@\let\@tempa\@tempb\fi% select the smaller of both
  \ifdim\@tempa\p@<\p@\scalebox{\@tempa}{\usebox\pandoc@box}%
  \else\usebox{\pandoc@box}%
  \fi%
}
% Set default figure placement to htbp
\def\fps@figure{htbp}
\makeatother





\setlength{\emergencystretch}{3em} % prevent overfull lines

\providecommand{\tightlist}{%
  \setlength{\itemsep}{0pt}\setlength{\parskip}{0pt}}



 


\KOMAoption{captions}{tableheading}
\makeatletter
\@ifpackageloaded{bookmark}{}{\usepackage{bookmark}}
\makeatother
\makeatletter
\@ifpackageloaded{caption}{}{\usepackage{caption}}
\AtBeginDocument{%
\ifdefined\contentsname
  \renewcommand*\contentsname{Table of contents}
\else
  \newcommand\contentsname{Table of contents}
\fi
\ifdefined\listfigurename
  \renewcommand*\listfigurename{List of Figures}
\else
  \newcommand\listfigurename{List of Figures}
\fi
\ifdefined\listtablename
  \renewcommand*\listtablename{List of Tables}
\else
  \newcommand\listtablename{List of Tables}
\fi
\ifdefined\figurename
  \renewcommand*\figurename{Figure}
\else
  \newcommand\figurename{Figure}
\fi
\ifdefined\tablename
  \renewcommand*\tablename{Table}
\else
  \newcommand\tablename{Table}
\fi
}
\@ifpackageloaded{float}{}{\usepackage{float}}
\floatstyle{ruled}
\@ifundefined{c@chapter}{\newfloat{codelisting}{h}{lop}}{\newfloat{codelisting}{h}{lop}[chapter]}
\floatname{codelisting}{Listing}
\newcommand*\listoflistings{\listof{codelisting}{List of Listings}}
\makeatother
\makeatletter
\makeatother
\makeatletter
\@ifpackageloaded{caption}{}{\usepackage{caption}}
\@ifpackageloaded{subcaption}{}{\usepackage{subcaption}}
\makeatother
\usepackage{bookmark}
\IfFileExists{xurl.sty}{\usepackage{xurl}}{} % add URL line breaks if available
\urlstyle{same}
\hypersetup{
  pdfauthor={Norah Jones},
  colorlinks=true,
  linkcolor={blue},
  filecolor={Maroon},
  citecolor={Blue},
  urlcolor={Blue},
  pdfcreator={LaTeX via pandoc}}


\author{Norah Jones}
\date{2025-08-31}
\begin{document}

\renewcommand*\contentsname{Table of contents}
{
\hypersetup{linkcolor=}
\setcounter{tocdepth}{2}
\tableofcontents
}

\bookmarksetup{startatroot}

\chapter*{Pengantar}\label{pengantar}
\addcontentsline{toc}{chapter}{Pengantar}

\markboth{Pengantar}{Pengantar}

\textbf{Selamat Datang di Mata Kuliah EL2007 Sinyal dan Sistem!}

Halo para mahasiswa, selamat datang di halaman awal mata kuliah Sinyal
dan Sistem. Halaman ini dirancang untuk menjadi panduan Anda dalam
menavigasi seluruh kegiatan pembelajaran selama satu semester ke depan.

{[}cite\_start{]}Tujuan utama dari mata kuliah ini adalah agar Anda
mampu menganalisis sifat-sifat sinyal dan sistem dalam berbagai domain
(waktu, frekuensi, dan Laplace) {[}cite: 4{]}{[}cite\_start{]},
merancang filter dan pengendali sederhana secara matematis {[}cite:
5{]}{[}cite\_start{]}, serta menggunakan perangkat lunak sebagai alat
bantu analisis{[}cite: 6{]}.

Untuk mencapai tujuan tersebut, kita akan mengikuti alur belajar
mingguan yang terstruktur. Mohon baca dan pahami alur kerja serta
petunjuk langkah demi langkah di bawah ini.

\begin{center}\rule{0.5\linewidth}{0.5pt}\end{center}

\section*{\texorpdfstring{\textbf{Workflow Belajar Mingguan
Anda}}{Workflow Belajar Mingguan Anda}}\label{workflow-belajar-mingguan-anda}
\addcontentsline{toc}{section}{\textbf{Workflow Belajar Mingguan Anda}}

\markright{\textbf{Workflow Belajar Mingguan Anda}}

Setiap minggunya, Anda akan mengikuti empat tahapan utama (P1-P4) untuk
setiap topik yang dibahas. Proses ini dirancang untuk membangun
pemahaman dari konsep dasar hingga aplikasi, sambil mendokumentasikannya
dalam Jurnal Belajar.

\begin{longtable}[]{@{}
  >{\raggedright\arraybackslash}p{(\linewidth - 6\tabcolsep) * \real{0.2500}}
  >{\raggedright\arraybackslash}p{(\linewidth - 6\tabcolsep) * \real{0.2500}}
  >{\raggedright\arraybackslash}p{(\linewidth - 6\tabcolsep) * \real{0.2500}}
  >{\raggedright\arraybackslash}p{(\linewidth - 6\tabcolsep) * \real{0.2500}}@{}}
\toprule\noalign{}
\begin{minipage}[b]{\linewidth}\raggedright
Tahap
\end{minipage} & \begin{minipage}[b]{\linewidth}\raggedright
Nama
\end{minipage} & \begin{minipage}[b]{\linewidth}\raggedright
Aktivitas Utama
\end{minipage} & \begin{minipage}[b]{\linewidth}\raggedright
Output
\end{minipage} \\
\midrule\noalign{}
\endhead
\bottomrule\noalign{}
\endlastfoot
\textbf{(P1)} & \textbf{Persiapan} & Membaca RPS dan Tujuan Belajar
mingguan. & Pemahaman tentang apa yang harus dicapai. \\
\textbf{(P2)} & \textbf{Studi Materi} & Mempelajari materi (slide/buku)
sambil membuat \textbf{Peta Dasar} (ringkasan konsep). & Sebuah Peta
Dasar (\emph{mind map}/catatan terstruktur). \\
\textbf{(P3)} & \textbf{Latihan Soal} & Mengerjakan soal latihan untuk
membangun \textbf{Peta Aplikasi} (koneksi konsep ke soal). & Laporan
berisi jawaban soal latihan. \\
\textbf{(P4)} & \textbf{Ujian Topik} & Menjawab soal ujian singkat
dengan memanfaatkan peta yang sudah disusun. & Jawaban ujian topik yang
diunggah. \\
\end{longtable}

\textbf{\emph{Jurnal Belajar}}: Di setiap tahap, Anda wajib mencatat
proses, kesulitan, dan pemahaman baru Anda dalam sebuah Jurnal Belajar.

\begin{center}\rule{0.5\linewidth}{0.5pt}\end{center}

\section*{\texorpdfstring{\textbf{How-To: Petunjuk Langkah Demi Langkah
(Contoh: Minggu
1)}}{How-To: Petunjuk Langkah Demi Langkah (Contoh: Minggu 1)}}\label{how-to-petunjuk-langkah-demi-langkah-contoh-minggu-1}
\addcontentsline{toc}{section}{\textbf{How-To: Petunjuk Langkah Demi
Langkah (Contoh: Minggu 1)}}

\markright{\textbf{How-To: Petunjuk Langkah Demi Langkah (Contoh: Minggu
1)}}

Berikut adalah panduan konkret tentang apa yang harus Anda lakukan
setiap minggu, menggunakan materi Minggu 1 sebagai contoh.

\section*{\texorpdfstring{\textbf{Langkah 1: Persiapan dan Orientasi
(P1)}}{Langkah 1: Persiapan dan Orientasi (P1)}}\label{langkah-1-persiapan-dan-orientasi-p1}
\addcontentsline{toc}{section}{\textbf{Langkah 1: Persiapan dan
Orientasi (P1)}}

\markright{\textbf{Langkah 1: Persiapan dan Orientasi (P1)}}

\begin{enumerate}
\def\labelenumi{\arabic{enumi}.}
\tightlist
\item
  \textbf{Buka Paket Bahan Kuliah Minggu 1}.
\item
  Baca bagian \textbf{(P1) Petunjuk Persiapan dan Tujuan Belajar}.
\item
  {[}cite\_start{]}\textbf{Identifikasi Topik Pekan Ini}: Anda akan tahu
  bahwa topik minggu ini adalah ``Deskripsi Matematis Sinyal Waktu
  Kontinu''{[}cite: 9{]}.
\item
  {[}cite\_start{]}\textbf{Pahami Tujuan Anda}: Tujuan utama Anda adalah
  ``Memahami dasar-dasar sinyal waktu kontinu dan representasi
  matematisnya''{[}cite: 10{]}.
\item
  {[}cite\_start{]}\textbf{Siapkan Rujukan}: Siapkan buku
  \textbf{Oppenheim Bab 2} dan \textbf{Schaum's Outline Bab 1} seperti
  yang tercantum pada rujukan{[}cite: 12, 13{]}.
\item
  \textbf{Catat di Jurnal Belajar}: Tulis tanggal, topik, dan tujuan
  belajar Anda untuk minggu ini.
\end{enumerate}

\section*{\texorpdfstring{\textbf{Langkah 2: Pelajari Materi dan Buat
Peta Dasar
(P2)}}{Langkah 2: Pelajari Materi dan Buat Peta Dasar (P2)}}\label{langkah-2-pelajari-materi-dan-buat-peta-dasar-p2}
\addcontentsline{toc}{section}{\textbf{Langkah 2: Pelajari Materi dan
Buat Peta Dasar (P2)}}

\markright{\textbf{Langkah 2: Pelajari Materi dan Buat Peta Dasar (P2)}}

\begin{enumerate}
\def\labelenumi{\arabic{enumi}.}
\tightlist
\item
  \textbf{Pelajari Materi Slide}: Buka dan pelajari materi slide di
  bagian \textbf{(P2) Materi dan Peta Dasar}.
\item
  \textbf{Buat Peta Dasar}: Sambil membaca, buatlah sebuah \emph{mind
  map} atau catatan terstruktur. Peta ini harus berisi:

  \begin{itemize}
  \tightlist
  \item
    Definisi sinyal.
  \item
    Klasifikasi sinyal (periodik/aperiodik, genap/ganjil, energi/daya).
  \item
    Gambar dan rumus sinyal-sinyal dasar (step, impuls, ramp).
  \item
    Aturan operasi dasar (pergeseran, penskalaan, pembalikan waktu).
  \end{itemize}
\item
  \textbf{Catat di Jurnal Belajar}: Tuliskan konsep-konsep kunci yang
  baru Anda pelajari dan bagian mana yang paling menantang.
\end{enumerate}

\section*{\texorpdfstring{\textbf{Langkah 3: Kerjakan Latihan dan Buat
Peta Aplikasi
(P3)}}{Langkah 3: Kerjakan Latihan dan Buat Peta Aplikasi (P3)}}\label{langkah-3-kerjakan-latihan-dan-buat-peta-aplikasi-p3}
\addcontentsline{toc}{section}{\textbf{Langkah 3: Kerjakan Latihan dan
Buat Peta Aplikasi (P3)}}

\markright{\textbf{Langkah 3: Kerjakan Latihan dan Buat Peta Aplikasi
(P3)}}

\begin{enumerate}
\def\labelenumi{\arabic{enumi}.}
\tightlist
\item
  \textbf{Buka Soal Latihan}: Akses daftar 20 soal latihan di bagian
  \textbf{(P3) Soal Latihan dan Peta Aplikasi}.
\item
  \textbf{Kerjakan Soal}: Coba kerjakan setiap soal dalam sebuah
  laporan.
\item
  \textbf{Buat Peta Aplikasi}: Untuk setiap soal, identifikasi:

  \begin{itemize}
  \tightlist
  \item
    ``Soal ini menguji konsep apa dari Peta Dasar saya?''
  \item
    ``Langkah-langkah apa yang saya perlukan untuk menyelesaikannya?''
  \item
    Hubungkan jenis soal dengan strategi penyelesaiannya. Inilah ``Peta
    Aplikasi'' Anda.
  \end{itemize}
\item
  \textbf{Selesaikan Laporan}: Pastikan laporan Anda berisi langkah
  pengerjaan yang jelas.
\item
  \textbf{Catat di Jurnal Belajar}: Soal nomor berapa yang paling sulit?
  Konsep mana yang perlu Anda pelajari lagi setelah mengerjakan latihan?
\end{enumerate}

\section*{\texorpdfstring{\textbf{Langkah 4: Uji Pemahaman dan Unggah
Hasil
(P4)}}{Langkah 4: Uji Pemahaman dan Unggah Hasil (P4)}}\label{langkah-4-uji-pemahaman-dan-unggah-hasil-p4}
\addcontentsline{toc}{section}{\textbf{Langkah 4: Uji Pemahaman dan
Unggah Hasil (P4)}}

\markright{\textbf{Langkah 4: Uji Pemahaman dan Unggah Hasil (P4)}}

\begin{enumerate}
\def\labelenumi{\arabic{enumi}.}
\tightlist
\item
  \textbf{Kerjakan Soal Ujian Topik}: Buka bagian \textbf{(P4) Soal
  Ujian Topik} dan kerjakan 3 soal yang tersedia.
\item
  \textbf{Manfaatkan Peta Anda}: Gunakan Peta Dasar dan Peta Aplikasi
  yang telah Anda buat sebagai alat bantu untuk menjawab soal dengan
  lebih cepat dan terstruktur.
\item
  \textbf{Unggah Jawaban}: Unggah hasil pekerjaan Anda ke platform yang
  telah ditentukan (GitHub, LMS, dll.).
\item
  \textbf{Catat di Jurnal Belajar}: Lakukan refleksi akhir. Apakah Anda
  berhasil mencapai tujuan belajar minggu ini? Apa yang akan Anda
  perbaiki untuk minggu depan?
\end{enumerate}

Selamat belajar, dan jangan ragu untuk berdiskusi jika menemui
kesulitan!

\bookmarksetup{startatroot}

\chapter{Introduction}\label{introduction}

.

\bookmarksetup{startatroot}

\chapter{Capaian Belajar}\label{capaian-belajar}

Berikut adalah rancangan pembelajaran satu semester untuk mata kuliah
\textbf{Sinyal dan Sistem (EL2007)}, yang telah ditambahkan dengan
rujukan pada buku teks \emph{Signals and Systems} oleh Oppenheim dan
\emph{Schaum's Outline of Signals and Systems} oleh Hwei Hsu.

\textbf{Rencana Pembelajaran Satu Semester}

\textbf{Kode Mata Kuliah:} EL2007 \textbf{Nama Mata Kuliah:} Sinyal dan
Sistem / Signals and Systems \textbf{Jumlah SKS:} 3 SKS
\textbf{Penyelenggara:} 132 - Teknik Elektro / STEI
\textbf{Co-requisite:} MA2074 Matematika Rekayasa IIA / Engineering
Mathematics IIA \textbf{Bahasa Pengantar:} Bahasa Indonesia / English

\textbf{Capaian Pembelajaran Mata Kuliah (CPMK):} Setelah mengikuti mata
kuliah ini, mahasiswa diharapkan mampu: 1. \textbf{Menganalisis sifat
sinyal dan sistem} dalam domain waktu, domain frekuensi, dan domain
Laplace. 2. \textbf{Merancang filter dan pengendali secara matematis}
pada studi kasus. 3. \textbf{Menggunakan alat bantu (perangkat lunak)}
untuk menganalisis sinyal dan sistem.

\textbf{Metode Pembelajaran:} Ceramah, Diskusi, Studi Kasus, Belajar
\textbf{Modalitas Pembelajaran:} Luring, Daring, Bauran (Offline,
Online, Hybrid) \textbf{Metode Penilaian:} PR (Pekerjaan Rumah), Kuis,
Ujian \textbf{Jenis Nilai:} ABCDE

\begin{center}\rule{0.5\linewidth}{0.5pt}\end{center}

\textbf{Distribusi Materi dan Kegiatan Per Minggu (14 Minggu
Perkuliahan):}

\textbf{Minggu 1:} * \textbf{Bahan Kajian:} Deskripsi Matematis Sinyal
Waktu Kontinu (Mathematical Description of Continuous-time Signals). *
\textbf{CPMK Terkait:} Memahami dasar-dasar sinyal waktu kontinu dan
representasi matematisnya. * \textbf{Kegiatan:} Ceramah dan Diskusi
tentang jenis-jenis sinyal dan operasi dasar sinyal. * \textbf{Rujukan:}
* \textbf{Oppenheim:} Bab 2 (Signals, Transformations of the Independent
Variable, Basic Continuous-Time Signals). * \textbf{Schaum's Series:}
Bab 1 (Signals and Systems, Signals and Classification of Signals, Basic
Continuous-Time Signals). Lihat juga Solved Problems seperti 1.3, 1.4,
1.5, 1.19, 1.46, 1.47, 1.48 untuk latihan.

\textbf{Minggu 2-3:} * \textbf{Bahan Kajian:} Deskripsi Sistem di Domain
Waktu: Persamaan Diferensial dan Respon Impuls (Time-Domain Description
of Systems: Differential Equation and Impulse Response). * \textbf{CPMK
Terkait:} Mengenal karakteristik sistem linear tak-berubah waktu (LTI)
dan representasinya melalui persamaan diferensial dan respon impuls. *
\textbf{Kegiatan:} Ceramah dan Diskusi mengenai konsep sistem,
sifat-sifatnya, dan pentingnya respon impuls. * \textbf{Penilaian (Akhir
Minggu 3):} PR 1 (terkait deskripsi sinyal dan sistem di domain waktu).
* \textbf{Rujukan:} * \textbf{Oppenheim:} Bab 2 (Systems, Properties of
Systems), Bab 3 (Linear Time-Invariant Systems, The Representation of
Signals in Terms of Impulses, Systems Described by Differential and
Difference Equations). * \textbf{Schaum's Series:} Bab 1 (Systems and
Classification of Systems), Bab 2 (Introduction, Systems Described by
Differential Equations, Eigenfunctions of Continuous-Time LTI Systems).
Solved Problems seperti 1.44, 1.45, 1.51, 1.52 relevan untuk fungsi
eigen dan deskripsi sistem.

\textbf{Minggu 4-5:} * \textbf{Bahan Kajian:} Analisis Sistem di Domain
Waktu: Solusi Persamaan Diferensial dan Konvolusi (Time-Domain Analysis
of Systems: Solution to Differential Equation and Convolution). *
\textbf{CPMK Terkait:} Mampu menganalisis respon sistem LTI menggunakan
konvolusi dan menyelesaikan persamaan diferensial yang menggambarkan
sistem. * \textbf{Kegiatan:} Ceramah, Diskusi, dan Studi Kasus tentang
aplikasi operasi konvolusi untuk mendapatkan respon sistem. *
\textbf{Rujukan:} * \textbf{Oppenheim:} Bab 3 (Discrete-Time LTI
Systems: The Convolution Sum, Continuous-Time LTI Systems: The
Convolution Integral, Systems Described by Differential and Difference
Equations). * \textbf{Schaum's Series:} Bab 2 (Response of a
Continuous-Time LTI System and the Convolution Integral, Systems
Described by Differential Equations). Lihat Solved Problems seperti 2.1,
2.8, 2.14, 2.46, 2.64 untuk latihan.

\textbf{Minggu 6-7:} * \textbf{Bahan Kajian:} Deskripsi Domain
Frekuensi: Transformasi Fourier (Frequency-Domain Description: Fourier
Transform). * \textbf{CPMK Terkait:} Memahami konsep Transformasi
Fourier sebagai alat untuk menganalisis sinyal dan sistem di domain
frekuensi. * \textbf{Kegiatan:} Ceramah, Diskusi, dan Belajar Mandiri
(latihan soal) tentang Transformasi Fourier untuk berbagai jenis sinyal.
* \textbf{Penilaian (Akhir Minggu 7):} Kuis 1 (tentang Transformasi
Fourier). * \textbf{Rujukan:} * \textbf{Oppenheim:} Bab 4
(Representation of Aperiodic Signals: The Continuous-Time Fourier
Transform, Properties of the Continuous-Time Fourier Transform). *
\textbf{Schaum's Series:} Bab 5 (Introduction, The Fourier Transform,
Properties of the Continuous-Time Fourier Transform). Lihat Solved
Problems seperti 5.31, 5.32, 5.35 untuk latihan.

\textbf{Minggu 8:} * \textbf{Bahan Kajian:} Analisis Sistem di Domain
Frekuensi: Respon Frekuensi (Frequency-Domain Analysis of Systems:
Frequency Response). * \textbf{CPMK Terkait:} Mampu menganalisis respon
frekuensi sistem dan memahami dampaknya terhadap sinyal input. *
\textbf{Kegiatan:} Ceramah, Diskusi, dan Studi Kasus mengenai
karakteristik filter di domain frekuensi. * \textbf{Rujukan:} *
\textbf{Oppenheim:} Bab 4 (The Frequency Response of Systems
Characterized by Linear Constant-Coefficient Differential Equations,
First-Order and Second-Order Systems). * \textbf{Schaum's Series:} Bab 5
(The Frequency Response of Continuous-Time LTI Systems). Lihat Solved
Problems seperti 5.45, 5.46, 5.47 untuk latihan.

\textbf{Minggu 9-10:} * \textbf{Bahan Kajian:} Deskripsi Domain Laplace:
Transformasi Laplace (Laplace-Domain Description: Laplace Transform). *
\textbf{CPMK Terkait:} Memahami Transformasi Laplace dan inversnya
sebagai alat yang ampuh untuk analisis sistem. * \textbf{Kegiatan:}
Ceramah, Diskusi, dan Belajar Mandiri (latihan soal) tentang
Transformasi Laplace untuk menyelesaikan persamaan diferensial. *
\textbf{Penilaian (Akhir Minggu 10):} PR 2 (terkait Transformasi
Laplace). * \textbf{Rujukan:} * \textbf{Oppenheim:} Bab 9 (Laplace and
z-Transforms: Introduction, Relationship to Fourier Transform, Poles and
Zeros, Region of Convergence of a Laplace or z-transform and its
relationship to properties of the signal with which it is associated;
inverse transforms using partial fraction expansion; basic transform
properties). * \textbf{Schaum's Series:} Bab 3 (The Laplace Transform,
Laplace Transforms of Some Common Signals, Properties of the Laplace
Transform, The Inverse Laplace Transform, The Unilateral Laplace
Transform). Lihat Solved Problems seperti 3.10, 3.15, 3.32, 3.33, 3.34,
3.37, 3.38 untuk latihan.

\textbf{Minggu 11-12:} * \textbf{Bahan Kajian:} Analisis Sistem di
Domain Laplace: Fungsi Alih (Laplace-Domain Analysis of System: Transfer
Function). * \textbf{CPMK Terkait:} Mampu menganalisis sistem
menggunakan konsep fungsi alih dan diagram blok di domain Laplace. *
\textbf{Kegiatan:} Ceramah, Diskusi, dan Studi Kasus analisis sistem
kompleks menggunakan fungsi alih. * \textbf{Penilaian (Akhir Minggu
12):} Kuis 2 (tentang Fungsi Alih). * \textbf{Rujukan:} *
\textbf{Oppenheim:} Bab 9 (System Functions for LTI Systems,
Interconnections of LTI Systems). * \textbf{Schaum's Series:} Bab 3 (The
System Function, Transform Circuits). Lihat juga ilustrasi sistem
kaskade dan paralel.

\textbf{Minggu 13:} * \textbf{Bahan Kajian:} Studi kasus: Desain Filter
(Case Study: Filter Designs). * \textbf{CPMK Terkait:} Mampu merancang
filter secara matematis dan menggunakan perangkat lunak untuk
memverifikasi desain. * \textbf{Kegiatan:} Studi Kasus dan Belajar
mandiri/praktikum menggunakan perangkat lunak untuk simulasi desain
filter (misalnya, low-pass, high-pass). * \textbf{Penilaian:} PR 3
(proyek desain filter). * \textbf{Rujukan:} * \textbf{Oppenheim:} Bab 6
(Introduction to Filtering in Continuous-Time and Discrete-Time, Ideal
Frequency-Selective Filters, Examples of Filters described by
differential and difference equations, Butterworth filters). *
\textbf{Schaum's Series:} Bab 5 (Filtering, Bandwidth), Bab 6 (System
Response to Sampled Continuous-Time Sinusoids, Simulation), lihat pula
bilinear transformation (Prob. 6.46, 6.47).

\textbf{Minggu 14:} * \textbf{Bahan Kajian:} Studi kasus: Pengantar
Sistem Kendali Linier Umpan Balik (Case Study: Introduction to Linear
Feedback Control System). * \textbf{CPMK Terkait:} Memahami prinsip
dasar sistem kendali umpan balik dan mampu menganalisis karakteristik
performanya. * \textbf{Kegiatan:} Ceramah, Diskusi, dan Studi Kasus
pengantar konsep sistem kendali. * \textbf{Rujukan:} *
\textbf{Oppenheim:} Bab 11 (Linear Feedback Systems: Introduction,
Linear Feedback Systems, Some Applications and Consequences of Feedback,
Root-Locus Analysis, The Nyquist Stability Criterion, Gain and Phase
Margins). * \textbf{Schaum's Series:} Bab 7 (State Space Analysis:
Introduction, The Concept of State, State Space Representation of
Discrete-Time LTI Systems, State Space Representation of Continuous-Time
LTI Systems, Solutions of State Equations for Discrete-Time LTI Systems,
Solutions of State Equations for Continuous-Time LTI Systems). (Meskipun
Bab 7 Schaum's lebih fokus pada analisis ruang keadaan, ini adalah dasar
untuk sistem kendali).

\textbf{Minggu Ujian Akhir Semester:} * \textbf{Penilaian:} Ujian Akhir
Semester (mencakup seluruh bahan kajian dari minggu 1 hingga 14).

\begin{center}\rule{0.5\linewidth}{0.5pt}\end{center}

\bookmarksetup{startatroot}

\chapter{Summary}\label{summary}

In summary, this book has no content whatsoever.

\bookmarksetup{startatroot}

\chapter*{References}\label{references}
\addcontentsline{toc}{chapter}{References}

\markboth{References}{References}

\phantomsection\label{refs}




\end{document}
